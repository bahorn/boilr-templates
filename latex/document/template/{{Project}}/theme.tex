\newcommand{\coursecode}{ {{CourseCode}} }
\newcommand{\coursename}{ {{CourseName}} }

\newcommand{\myauthor}{ {{Author}} }
\newcommand{\projectname}{ {{Title}} }
\newcommand{\mytitle}{\coursecode - \projectname}
\newcommand{\currentdate}{ {{Date}} }

\usepackage{xspace}
\usepackage[style=ieee,backend=biber]{biblatex}
\usepackage{fancyhdr}
\usepackage{graphicx}
\usepackage{amsmath,amssymb,amsthm}
\usepackage{enumitem}
\usepackage{xcolor}
\usepackage{hyperref}
\usepackage{listings}
\usepackage{totcount}
\setcounter{biburllcpenalty}{7000}
\setcounter{biburlucpenalty}{8000}
\usepackage[a4paper, margin=1in]{geometry}

\graphicspath{ {./figures/} }

%--------------------------------------------------------------------------------
% Style stuff
\pagestyle{fancy}
\lhead{\coursename\ (\coursecode)}
\chead{}
\rhead{\mytitle}
\lfoot{\coursecode}
\cfoot{}
\rfoot{\thepage}
\renewcommand{\headrulewidth}{0.4pt}
\renewcommand{\footrulewidth}{0.4pt}
\renewcommand{\familydefault}{\sfdefault}


\newcommand{\withpoints}[1]{%
  \addtocounter{pointscounter}{#1} \printpoints{#1}
}
\newcommand{\printpoints}[1]{%
   \ifthenelse{#1 = 0}
              {}
              {\textit{(#1 points)}}\mbox{}\\
}

\newenvironment{questions}{\begin{enumerate}[label=\textbf{Question \arabic*}
                                            ,labelwidth=!
                                            ,align=left
                                            ]}
                          {\end{enumerate}}
\newenvironment{parts}{\begin{enumerate}[leftmargin=*
                                        ]}
                      {\end{enumerate}}

\newcommand{\question}[1][0]{\item\stepcounter{questionscounter}\withpoints{#1}}
\newcommand{\qpart}[1][0]{\item \withpoints{#1}}

\newtotcounter{questionscounter}
\newtotcounter{pointscounter}

\newcommand{\numquestions}{\total{questionscounter}}
\newcommand{\numpoints}{\total{pointscounter}}

%--------------------------------------------------------------------------------
\newcommand{\myremark}[3]{\textcolor{blue}{\textsc{#1 #2:}} \textcolor{red}{\textsf{#3}}}
% \renewcommand{\myremark}[3]{}
\newcommand{\frank}[2][says]{\myremark{Frank}{#1}{#2}}

%--------------------------------------------------------------------------------
% Theorem Environments
\newtheorem{theorem} {Theorem}
\newtheorem{lemma}[theorem] {Lemma}
\newtheorem{corollary}[theorem] {Corollary}
\newtheorem{problem}[theorem] {Problem}
\newtheorem{observation}[theorem] {Observation}
\newtheorem{claim}[theorem] {Claim}
\newtheorem{invariant}[theorem] {Invariant}

%--------------------------------------------------------------------------------
% Otherwise useful Macro's
\newcommand{\eps}{\ensuremath{\varepsilon}\xspace}
\DeclareMathOperator{\argmin}{argmin}

\newcommand{\mkmbb}[1]{\ensuremath{\mathbb{#1}}\xspace}
\newcommand{\R}{\mkmbb{R}}

\newcommand{\mkmcal}[1]{\ensuremath{\mathcal{#1}}\xspace}
\newcommand{\RR}{\mkmcal{R}}
